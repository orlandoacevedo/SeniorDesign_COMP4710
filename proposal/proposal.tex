\documentstyle[11pt]{article}
\setlength{\topmargin}{-.8in}
\setlength{\textheight}{9in}
\setlength{\oddsidemargin}{.125in}
\setlength{\textwidth}{6.5in}
\begin{document}
\title{Spring 2012 Senior Design Proposal:\\
GPU Computing}
\author{Riley Spahn}
\renewcommand{\today}{December 2011}
\maketitle

\section*{Project Goal}
We will be using nVidia CUDA to develop a massively parallel Monte Carlo 
simulation of molecular dynamics based on Dr. Acevedo's legacy Fortran code.
The simulation we develop will be a framework that Dr. Acevedo and the chemistry
department can use as a basis to easily develop future simulations.  After we
have developed the simulation we will benchmark our simulation against Dr.
Acevedo's legacy Fortran and compare the speed up from versus potential loss of
accuracy.  

\section*{Learning Material}
There is ample learning material available from nVidia and other sources online
and in print.  Our best reference for learning to use CUDA is the CUDA
Programming Guide.  It gives complete information about the programming model,
memory hierarchy, multiple device execution, etc.
Two potentially useful printed books are \emph{CUDA by Example} by Sanders and
Kandrot and \emph{Programming Massively Parallel Processors} by Kirk and Hwu.
\emph{GPU Gems} parts one, two and three are freely available online.  These
three books focus more on using the GPU for graphics but they still have the
potential to be useful.

\section*{Tool Chain}
We will use the CUDA toolkit provided by nVidia.  The toolkit provides many
libraries with functions and data structures that are optimized for parallel
execution.  We will develop using C/C++.  All group members should at least be
familiar with C++ since it is required earlier in the curriculum.  nVidia's
toolkit includes a C/C++ compiler (NVCC).  The toolkit also includes a profiler,
memory debugging tool as well as the console based CUDA-GDB Debugger.

\section*{Hardware}
Common hardware will be required for this project.  CUDA is compatible with
nVidia's consumer GeForce GPUs with at least 256 MB of memory as well as with
their professional Quadro and
Tesla products.  We will most likely use one of the GeForce products because
they are the most common and should be readily available. A full list of CUDA
compatible graphics cards can be found on nVidia's website. Ideally we would like
to use a headless Linux machine that we can access through SSH or other means.
Without a monitor the graphics can devote more time to our computations and less
to display.  We would like to use a Linux machine because the CUDA-GDB looks to be
the most viable debugging tool and is only available on Linux.

\end{document}
